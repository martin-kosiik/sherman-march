\begin{table}[!h]

\caption{\label{tab:other_out_ols_diff_treat_def}Other outcomes - OLS - different treatment definitions}
\centering
\begin{tabular}[t]{llllll}
\toprule
 & First names (1880) & First names (1930) & Street names & Monuments & Lynch rate\\
\midrule
Sherman's march (10 miles) & -0.017 & 0.210 & 0.126 & 0.018 & 0.126\\
 & (0.020) & (0.080) & (0.065) & (0.059) & (0.065)\\
Sherman's march (20 miles) & 0.004 & 0.186 & 0.134 & -0.004 & 0.134\\
 & (0.020) & (0.082) & (0.079) & (0.057) & (0.079)\\
Sherman's march (50 miles) & 0.007 & -0.042 & 0.055 & 0.080 & 0.055\\
 & (0.022) & (0.103) & (0.078) & (0.061) & (0.078)\\
\bottomrule
\multicolumn{6}{l}{\textsuperscript{a} The standard set of controls was included in all specifications. The standard errors based on}\\
\multicolumn{6}{l}{HC2 variance estimator are in the parentheses. First names (1880) is the share (in \%) of}\\
\multicolumn{6}{l}{whites born after 1865 in a given county with first name that we classified as Confederate in}\\
\multicolumn{6}{l}{1880 census. First names (1930) is defined the same, only 1930 census is used. Street names is}\\
\multicolumn{6}{l}{the share (in \%) of streets and roads in a given that contain a surname of a Confederate}\\
\multicolumn{6}{l}{figure in their name. Monuments is a an indicator variable that equals one if a Confederate}\\
\multicolumn{6}{l}{monument was present in a county in 2019. Lynch rate is defined as the number of lynchings in a}\\
\multicolumn{6}{l}{county from 1882 to 1929 per 10,000,000 residents.}\\
\end{tabular}
\end{table}